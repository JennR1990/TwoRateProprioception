\documentclass[]{article}
\usepackage{lmodern}
\usepackage{amssymb,amsmath}
\usepackage{ifxetex,ifluatex}
\usepackage{fixltx2e} % provides \textsubscript
\ifnum 0\ifxetex 1\fi\ifluatex 1\fi=0 % if pdftex
  \usepackage[T1]{fontenc}
  \usepackage[utf8]{inputenc}
\else % if luatex or xelatex
  \ifxetex
    \usepackage{mathspec}
  \else
    \usepackage{fontspec}
  \fi
  \defaultfontfeatures{Ligatures=TeX,Scale=MatchLowercase}
\fi
% use upquote if available, for straight quotes in verbatim environments
\IfFileExists{upquote.sty}{\usepackage{upquote}}{}
% use microtype if available
\IfFileExists{microtype.sty}{%
\usepackage{microtype}
\UseMicrotypeSet[protrusion]{basicmath} % disable protrusion for tt fonts
}{}
\usepackage[margin=1in]{geometry}
\usepackage{hyperref}
\hypersetup{unicode=true,
            pdftitle={Two-Rate Proprioception - How Quickly Does Proprioception Recalibrate And Is It Related To The Two-Rate Model Of Motor Learning},
            pdfauthor={Jennifer E. Ruttle; Bernard Marius 't Hart; Denise Y.P. Henriques},
            pdfborder={0 0 0},
            breaklinks=true}
\urlstyle{same}  % don't use monospace font for urls
\usepackage{graphicx,grffile}
\makeatletter
\def\maxwidth{\ifdim\Gin@nat@width>\linewidth\linewidth\else\Gin@nat@width\fi}
\def\maxheight{\ifdim\Gin@nat@height>\textheight\textheight\else\Gin@nat@height\fi}
\makeatother
% Scale images if necessary, so that they will not overflow the page
% margins by default, and it is still possible to overwrite the defaults
% using explicit options in \includegraphics[width, height, ...]{}
\setkeys{Gin}{width=\maxwidth,height=\maxheight,keepaspectratio}
\IfFileExists{parskip.sty}{%
\usepackage{parskip}
}{% else
\setlength{\parindent}{0pt}
\setlength{\parskip}{6pt plus 2pt minus 1pt}
}
\setlength{\emergencystretch}{3em}  % prevent overfull lines
\providecommand{\tightlist}{%
  \setlength{\itemsep}{0pt}\setlength{\parskip}{0pt}}
\setcounter{secnumdepth}{0}
% Redefines (sub)paragraphs to behave more like sections
\ifx\paragraph\undefined\else
\let\oldparagraph\paragraph
\renewcommand{\paragraph}[1]{\oldparagraph{#1}\mbox{}}
\fi
\ifx\subparagraph\undefined\else
\let\oldsubparagraph\subparagraph
\renewcommand{\subparagraph}[1]{\oldsubparagraph{#1}\mbox{}}
\fi

%%% Use protect on footnotes to avoid problems with footnotes in titles
\let\rmarkdownfootnote\footnote%
\def\footnote{\protect\rmarkdownfootnote}

%%% Change title format to be more compact
\usepackage{titling}

% Create subtitle command for use in maketitle
\newcommand{\subtitle}[1]{
  \posttitle{
    \begin{center}\large#1\end{center}
    }
}

\setlength{\droptitle}{-2em}

  \title{Two-Rate Proprioception - How Quickly Does Proprioception Recalibrate
And Is It Related To The Two-Rate Model Of Motor Learning}
    \pretitle{\vspace{\droptitle}\centering\huge}
  \posttitle{\par}
    \author{Jennifer E. Ruttle \\ Bernard Marius 't Hart \\ Denise Y.P. Henriques}
    \preauthor{\centering\large\emph}
  \postauthor{\par}
    \date{}
    \predate{}\postdate{}
  

\begin{document}
\maketitle

\section{Overview}\label{overview}

We have previously found that proprioception recalibrates within 6
trials, whereas visuomotor adaptation takes longer to saturate
{[}\protect\hyperlink{ref-Ruttle2016}{1},\protect\hyperlink{ref-Ruttle2018}{2}{]}.

In this project we set out to investigate 1) exactly how quickly
proprioception recalibrates, 2) to disentangle it from the updating of
predicted sensory consequences, and 3) to see if it matches implicit
learning that should follow the slow process
{[}\protect\hyperlink{ref-McDougle2015}{3}{]} in the two-rate model
{[}\protect\hyperlink{ref-Smith2006}{4}{]}.

\subsection{Trial types}\label{trial-types}

To do all this we use a trial-by-trial approach, that is: one
reach-training trial is followed by one of various ``measurement''
trials: passive localization, active localization or no-cursor reaches
(all of them explained below). As a control we also had people pause for
a short bit instead. There wer four groups of participants and each did
one of the four ``measurement'' trials exclusively. With one exception,
the pause and no-cursor group also did a set of localization trials
before and after the regular paradigm also done by the other groups.

\begin{itemize}
\tightlist
\item
  reach training
\item
  passive localization (only proprioception)
\item
  active localization (proprioception + prediction)
\item
  no-cursor reach (classic measure of motor learning)
\item
  pause (control)
\item
  error-clamp
\end{itemize}

By comparing the trial-by-trial changes in the four paradigms, we can
investigate the time-course of all these processes in more depth than we
previously have.

\subsection{Figures}\label{figures}

Here is a list of planned figures with their intended meaning:

\begin{itemize}
\tightlist
\item
  Fig 1: A) setup, B) reach training + targets, C) localization +
  points, D) no-cursor + targets (illustrate the hardware and trial
  types)
\item
  Fig 2: paradigm, four phases (aligned, rotated, reversed,
  error-clamped), plus extra localization phases (illustrate the main
  paradigm\ldots{} add expected model fit?)
\item
  Fig 3: A,B,C,D) one subplot for each group showing reach deviations in
  the four main phases: individual traces, 95\% confidence intervals,
  median (show quality of the data)
\item
  Fig 4: A,B,C,D) one subplot for each group showing the median reach
  deviations in the four main phases, plus the two-rate model fit and
  it's fast and slow process (show quality of the fits)
\item
  Fig 5: comparing reach deviations over time, (show differences between
  the groups, or should this go in a previous figure? seems it would get
  cluttered)
\item
  Fig 6: A) localization throughout for the two localization groups, B)
  localization shift in the pause and no-cursor group, C) speed of
  localization shift, D/E) localization shift with the two processes for
  each group (show the incredible speed of localization shifts and how
  non-matching the slow but also the fast process are)
\item
  Fig 7: alternative models of localization? (20\% model works
  reasonably well\ldots{} even better than giving localization it's own
  two- or one-rate model)
\item
  Fig 8: A) time course of no-cursor changes (no-cursor could reflect a
  watered down version of the slow process?)
\end{itemize}

\subsection{Two-Rate model}\label{two-rate-model}

In the two-rate model, the motor output X at trial t, \(X_t\) is simply
the added output of the slow and fast process at trial t:

\begin{equation}
\label{tworate-total}
X_t = S_t + F_t
\end{equation}

The output of the slow process on the next trial (t+1) is determined by
the error on the previous trial multiplied by a learning rate L, and the
output on the previous trial multiplied by a retention rate R:

\begin{equation}
\label{tworate-slow}
S_{t+1} = (L^s \cdot E_t) + (R^s \cdot S_t)
\end{equation}

Similarly, the fast process has it's own learning rate and retention
rate:

\begin{equation}
\label{tworate-fast}
F_{t+1} = (L^f \cdot E_t) + (R^f \cdot F_t)
\end{equation}

The four parameters \(L^s\), \(R^s\), \(L^f\) and \(R^f\) will be fit to
match the data optimally according to a least-squared error algorithm.
They should each fall within the range {[}0,1{]}. To ensure that the
fast process learns faster than the slow process but also forgets faster
than the slow process the model is further constrained by requiring that
\(L^s < L^f\) and \(R^s > R^f\).

\section*{References}\label{references}
\addcontentsline{toc}{section}{References}

\hypertarget{refs}{}
\hypertarget{ref-Ruttle2016}{}
1. Ruttle JE, Cressman EK, 't Hart BM, Henriques DYP. Time course of
reach adaptation and proprioceptive recalibration during visuomotor
learning. PLOS ONE. Public Library of Science; 2016;11: 1--16.
doi:\href{https://doi.org/10.1371/journal.pone.0163695}{10.1371/journal.pone.0163695}

\hypertarget{ref-Ruttle2018}{}
2. Ruttle JE, 't Hart BM, Henriques DYP. The fast contribution of
visual-proprioceptive discrepancy to reach aftereffects and
proprioceptive recalibration. PLOS ONE. Public Library of Science;
2018;13: 1--16.
doi:\href{https://doi.org/10.1371/journal.pone.0200621}{10.1371/journal.pone.0200621}

\hypertarget{ref-McDougle2015}{}
3. McDougle SD, Bond KM, Taylor JA. Explicit and implicit processes
constitute the fast and slow processes of sensorimotor learning. Journal
of Neuroscience. Society for Neuroscience; 2015;35: 9568--9579.
doi:\href{https://doi.org/10.1523/JNEUROSCI.5061-14.2015}{10.1523/JNEUROSCI.5061-14.2015}

\hypertarget{ref-Smith2006}{}
4. Smith MA, Ghazizadeh A, Shadmehr R. Interacting adaptive processes
with different timescales underlie short-term motor learning. PLOS
Biology. Public Library of Science; 2006;4.
doi:\href{https://doi.org/10.1371/journal.pbio.0040179}{10.1371/journal.pbio.0040179}


\end{document}
